%
% Library Document for ACM/ICPC
% based on tex by kaicho-@IHI
% $Id$
%

\documentclass{jsarticle}
\usepackage{amsmath}
\usepackage{eclbkbox}
\usepackage{fancybox}
\usepackage{pifont}
\usepackage{txfonts}
\usepackage{picins}
\usepackage{ascmac}
\usepackage{graphicx}
\everymath{\displaystyle}
\newenvironment{code}%
 {\VerbatimEnvironment\begin{breakbox}\vspace{-1em}\footnotesize\begin{Verbatim}}%
 {\end{Verbatim}\vspace{-.8em}\end{breakbox}}
\addtolength{\oddsidemargin}{-.5cm}
\addtolength{\voffset}{-1.2cm}
\addtolength{\textwidth}{1cm}
\addtolength{\textheight}{2cm}
\setlength{\fboxsep}{.7em}
\renewcommand{\breakboxskip}{.7em}
\renewcommand{\ttdefault}{pcr}
\renewcommand{\vec}[1]{\boldsymbol{#1}}
\newcommand{\accepted}[1]{\ding{52} Accepted (#1)}
\newcommand{\nottested}[0]{\ding{56} NOT TESTED!}
\newcommand{\derived}[1]{\ding{224} Derived from #1}
\newcommand{\todo}[1]{\textbf{TODO: #1}}
\newcommand{\mymaketitle}[3]%
 {\begin{center}\begin{tabular}{lr}\textsf{\huge #1}\hspace{3.5cm}\mbox{}&#3\\\hline&#2\end{tabular}\end{center}}
\DeclareMathOperator{\vers}{vers}
\setcounter{tocdepth}{3}
\begin{document}

\mymaketitle{ACM/ICPC Library}{by qoo/CLAGGANO}{Revision: 2 (2006/12/06)}
\tableofcontents
\newpage

\section{��b}

\subsection{�e���v���[�g}

\begin{code}
#include <iostream>
#include <sstream>
#include <string>
#include <vector>
#include <deque>
#include <queue>
#include <set>
#include <map>
#include <algorithm>
#include <iterator>
#include <functional>
#include <utility>
#include <numeric>
#include <complex>
#include <cstdio>
#include <cmath>
#include <cctype>
#include <cassert>
using namespace std;

#define REP(i,n) for(int i = 0; i < (int)(n); i++)
#define FOR(i,c) for(__typeof((c).begin()) i = (c).begin(); i != (c).end(); ++i)
#define ALLOF(c) ((c).begin()), ((c).end())
#define IN(i,l,u) ((l) <= (i) && (i) < (u))

ifstream cin("_.txt");

int main() {
    return 0;
}

// Powered by Fate Testarossa
\end{code}



\subsection{������r}

\begin{code}
#define EP 1.0e-10
inline int SGN(double a) { return abs(a) < EP ? 0 : a > 0 ? 1 : -1; }
#define EQ(a,b) (SGN((a)-(b)) == 0)    // equal
#define NE(a,b) (SGN((a)-(b)) != 0)    // not equal
#define LE(a,b) (SGN((a)-(b)) <= 0)    // less or equal
#define LT(a,b) (SGN((a)-(b)) <  0)    // less
#define GE(a,b) (SGN((a)-(b)) >= 0)    // greater or equal
#define GT(a,b) (SGN((a)-(b)) >  0)    // greater
\end{code}







\section{�O���t}

\subsection{��{�f�[�^�\��}

\begin{code}
typedef double Weight;
const Weight WEIGHT_INFTY = numeric_limits<Weight>::max() / 4;
struct Edge {
    int src, dest;
    Weight weight;
};
bool operator>(const Edge& a, const Edge& b) {
    return (a.weight > b.weight);
}
typedef vector<Edge> Edges;
typedef vector<Edges> Graph; // �W���I�ȏd�݂‚��O���t�̗אڃ��X�g�\��
typedef vector< vector<Weight> > WMatrix; // �d�ݍs��
typedef vector< vector<int> > AList;      // �d�݂Ȃ��אڃ��X�g
\end{code}

\subsection{��ΑS�ŒZ�o�H(Dijkstra)}

\accepted{UVA10525 NEW TO BANGLADESH?}

$O(E \log V)$ �B
�p�X�̏d�݂͔񕉂łȂ���΂Ȃ�Ȃ��B

\begin{code}
Graph g;
vector<int> trace;

Weight shortest(int start, int goal) {
    int n = g.size();
    trace.assign(n, -2); // UNREACHABLE

    priority_queue<Edge, vector<Edge>, greater<Edge> > q;
    q.push((Edge){-1, start, 0}); // TERMINAL

    while(!q.empty()) {
        Edge e = q.top();
        q.pop();
        if (trace[e.dest] >= 0)
            continue;
        trace[e.dest] = e.src;
        if (e.dest == goal)
            return e.weight;
        FOR(it, g[e.dest])
            if (trace[it->dest] == -2)
                q.push((Edge){it->src, it->dest, e.weight + it->weight});
    }
    return -1;
}

vector<int> buildRoute(int goal) {
    if (trace[goal] == -2)
        return vector<int>(); // UNREACHABLE
    vector<int> route;
    for(int i = goal; i >= 0; i = trace[i])
        route.push_back(i);
    reverse(route.begin(), route.end());
    return route;
}
\end{code}

\subsection{�S�ΑS�ŒZ�o�H(Warshall-Floyd)}

\accepted{UVAArchive3569 World Finals 2005 Degrees of Separation}

$O(V^3)$ �B
\verb|WEIGHT_INFTY|�̃I�[�o�[�t���[���N����Ȃ��悤�ɋC���‚��邱�ƁB

\begin{code}
WMatrix g;
vector< vector<int> > trace;

void shortest() {
    int n = g.size();
    trace.assign(n, vector<int>(n, -1));
    REP(j, n) REP(i, n) REP(k, n)
        if (g[i][k] > g[i][j] + g[j][k])
            g[i][k] = g[i][j] + g[j][k] ,
                trace[i][k] = j;
}
void buildRoute(vector<int>& route, int src, int dest, bool rec = false) {
    if (!rec)
        route.clear();
    int inter = trace[src][dest];
    if (inter < 0)
        route.push_back(src);
    else
        buildRoute(route, src, inter, true) ,
            buildRoute(route, inter, dest, true);
    if (!rec)
        route.push_back(dest);
}
\end{code}


\subsection{��ΑS�ŒZ�o�H(Bellman-ford)}

\nottested

$O(VE)$ �B
�}�̏d�݂����ł���ꍇ�Ɏg���B���̃T�C�N��������ꍇ��\verb|false|��Ԃ��B

\begin{code}
Graph g;

bool shortest(int start, vector<Weight> costs, vector<int> trace) {
    int n = g.size();
    costs.assign(n, WEIGHT_INFTY);
    trace.assign(n, -2);  // UNREACHABLE
    costs[start] = 0;
    trace[start] = -1;    // TERMINAL

    REP(k, n) REP(a, n) {
        if (costs[a] != WEIGHT_INFTY) {
            FOR(it, g[a]) {
                int b = it->dest;
                Weight w = costs[a] + it->weight;
                if (w < costs[b]) {
                    costs[b] = w;
                    trace[b] = a;
                }
            }
        }
    }
    REP(a, n) FOR(it, g[a]) // negative cycle�̃`�F�b�N
        if (costs[a] + it->weight < costs[it->dest])
            return false;
    return true;
}
\end{code}

\subsection{�I�����C�����l�ޔ��� (Union-Find)}

\accepted{UVA10600 ACM CONTEST AND BLACKOUT}

\verb|-data[find_root(i)]|�̓m�[�h\verb|i|��������W���̑傫���ɂȂ��Ă���B

\begin{code}
int n;
vector<int> data(n, -1);
bool link(int a, int b) { // �V���ȕ������s����true
    int ra = find_root(a);
    int rb = find_root(b);
    if (ra != rb) {
        if (data[rb] < data[ra])
            swap(ra, rb);
        data[ra] += data[rb];
        data[rb] = ra;
    }
    return (ra != rb);
}
bool check(int a, int b) { // �����W���Ȃ�true
    return (find_root(a) == find_root(b));
}
int find_root(int a) { // ��\����Ԃ�
    return ((data[a] < 0) ? a : (data[a] = find_root(data[a])));
}
\end{code}

\subsection{�ŏ��S���(Prim)}

\accepted{UVA10034 Freckles}

�m�[�h0��������A�������̍ŏ��S��؂����߂�B�߂�l��\verb|(�؂̏d��, ��)|�B

\begin{code}
pair<Weight, Edges> prim(Graph& g) {
    int n = g.size();
    priority_queue<Edge, vector<Edge>, greater<Edge> > q;
    vector<bool> visited(n, false);
    Edges tree;
    Weight weight = 0;

    q.push((Edge){-1, 0, 0});
    while(!q.empty()) {
        Edge e = q.top();
        q.pop();
        if (visited[e.dest])
            continue;
        visited[e.dest] = true;
        if (e.src >= 0) {
            tree.push_back(e);
            weight += e.weight;
        }
        FOR(it, g[e.dest])
            if (!visited[it->dest])
                q.push(*it);
    }
    return make_pair(weight, tree);
}
\end{code}


\subsection{�ŏ��S��X(Kruskal)}

\accepted{UVA10600 ACM CONTEST AND BLACKOUT}

\todo{priority queue�͕K�v�Ȃ�}

$O(E \log E)$ �B
�ŏ��S��X�����߂�B�vUnionFind�B�߂�l��\verb|(�X�̏d��, �X)|�B
Union-Find���K�v�B

\begin{code}
pair<Weight, Edges> kruskal(Graph& g) {
    int n = g.size();
    priority_queue<Edge, vector<Edge>, greater<Edge> > q;
    UnionFind uf(n);
    REP(a, n) FOR(it, g[a])
        if (a < it->dest)
            q.push(*it);
    Edges tree;
    Weight weight = 0;
    while(!q.empty() && (int)tree.size() < n-1) {
        Edge e = q.top();
        q.pop();
        if (uf.link(e.src, e.dest))
            tree.push_back(e) ,
                weight += e.weight;
    }
    return make_pair(weight, tree);
}
\end{code}



% ��������

% \subsection{�g�|���W�J���\�[�g}

% \nottested

% \begin{code}
% Graph g;
% vector<int> order;
% vector<bool> visited;

% void sort() {
%     int n = g.size();
%     order.clear();
%     visited.assign(n, false);
%     REP(a, n) dfs(a);
% }
% void dfs(int a) {
%     if (visited[a])
%         return;
%     visited[a] = true;
%     FOR(it, g[a]) dfs(it->dest);
%     order.push_back(a);
% }
% \end{code}


\subsection{�񕔃O���t�̍ő�}�b�`���O}

\accepted{UVA10080 GopherII}

\derived{IHI Library}

$O(mn^2)$ �B�ő�}�b�`���O�̑傫���͍ŏ��_�핢�̑傫���Ɉ�v����B

\begin{code}
int n, m;                 // ���E�̃m�[�h��
vector< vector<int> > g;  // �O���t (left-index -> [right-index])
vector<int> z;            // �}�b�`���O (right-index -> left-index)
vector<bool> v;           // visited flag

BGM(int n, int m) : n(n), m(m), g(n) {}
void add_edge(int a, int b) {
    g[a].push_back(b);
}
int match() {
    z.assign(m, -1);
    int matches = 0;
    REP(a, n) {
        v.assign(m, false);
        if (augment(a))
            matches++;
    }
    return matches;
}
bool augment(int a) {
    if (a < 0),
        return true;
    vector<int>& e = g[a];
    REP(i, e.size()) {
        int b = e[i];
        if (!v[b]) {
            v[b] = true;
            if (augment(z[b])) {
                z[b] = a;
                return true;
            }
        }
    }
    return false;
}
\end{code}



\subsection{�񕔃O���t�̕ӍʐF}

\accepted{UVA10615 Rooks}

\begin{code}
int n, m;                 // ���E�̃m�[�h��
vector< vector<int> > g;  // �O���t (index -> [index])
vector< vector<int> > cl; // �F�� (index -> color -> index)

BGEC(int n, int m) : n(n), m(m), g(n+m) {}
void add_edge(int a, int b) { // ��a�A�Eb�̊Ԃɕӂ𒣂�
    g[a].push_back(n+b);
    g[n+b].push_back(a);
}
int color() {
    int d = 0; // �ʐF��
    REP(i, n+m)
        d = max<int>(d, g[i].size());
    cl.assign(n+m, vector<int>(d, -1));
    REP(a, n) REP(i, g[a].size()) {
        int b = g[a][i];
        int ca = min_element(ALLOF(cl[a])) - cl[a].begin();
        int cb = min_element(ALLOF(cl[b])) - cl[b].begin();
        if (ca != cb) {
            augment(b, ca, cb);
            cb = ca;
        }
        cl[a][ca] = b;
        cl[b][cb] = a;
    }
    return d;
}
void augment(int a, int c1, int c2) {
    int b = cl[a][c1];
    if (b >= 0) {
        augment(b, c2, c1);
        cl[b][c2] = a;
        cl[a][c1] = -1;
    }
    cl[a][c2] = b;
}
\end{code}


\subsection{�ő嗬(Edmonds-Karp)}

\accepted{UVA10330 Power Transmitter}

\derived{IHI Library}

\begin{code}
WMatrix g;

#define RESIDUE(from,to) (g[from][to] - (f[from][to] - f[to][from]))
Weight max_flow(int s, int t) {
    int n = g.size();
    WMatrix f(n, vector<Weight>(n, 0));
    Weight flow = 0;
    while(true) {
        queue<int> q;
        q.push(s);
        vector<int> trace(n, -1);
        trace[s] = s;
        while(!q.empty() && trace[t] < 0) {
            int i = q.front();
            q.pop();
            REP(j, n)
                if (trace[j] < 0 && RESIDUE(i, j) > 0)
                    trace[j] = i ,
                        q.push(j);
        }
        if (trace[t] < 0)
            break;
        Weight w = WEIGHT_INFTY;
        for(int j = t; trace[j] != j; j = trace[j])
            w = min(w, RESIDUE(trace[j], j));
        for(int j = t; trace[j] != j; j = trace[j])
            f[trace[j]][j] += w;
        flow += w;
    }
    return flow;
}
\end{code}


\subsection{���A����������}

\nottested

\derived{IHI Library}

\begin{code}
Graph g;
vector<bool> visited;

vector< vector<int> > scc() {
    int n = g.size();
    Graph r(n);   // make reversed graph
    REP(a, n) FOR(it, g[a])
        r[it->dest].push_back((Edge){it->dest, it->src});
    vector<int> order;
    visited.assign(n, false);
    REP(i, n) dfs(i, order);
    reverse(order.begin(), order.end());
    g.swap(r);
    vector< vector<int> > components;
    visited.assign(n, false);
    REP(i, n)
        if (!visited[order[i]])
            components.push_back(vector<int>()) ,
                dfs(order[i], components.back());
    g.swap(r);
    return components;
}
void dfs(int a, vector<int>& order) {
    if (visited[a])
        return;
    visited[a] = true;
    FOR(it, g[a]) dfs(it->dest, order);
    order.push_back(a);
}
\end{code}

\subsection{�֐ߓ_}

\nottested

\todo{��d�A����������������}

�߂�l�͊e�m�[�h����菜���ꂽ�Ƃ��ɐV���ɂł��镪��̐��ɑΉ�����B

\begin{code}
AList g;
vector<int> depths;
vector<int> counts;

void count() {
    int n = g.size();
    depths.assign(n, -1);
    counts.assign(n, 0);
    REP(a, n) {
        if (depths[a] >= 0)
            continue;
        dfs(a, 0);
        if (counts[a] > 0)
            counts[a]--;
    }
}
int dfs(int a, int cur) {
    depths[a] = cur;
    int upper = cur;
    FOR(it, g[a]) {
        int b = *it;
        int u = depths[b];
        if (u < 0) {
            u = dfs(b, cur+1);
            if (u >= cur)
                counts[a]++;
        }
        upper = min(upper, u);
    }
    return upper;
}
\end{code}

\subsection{�I�C���[�H}

\accepted{UVA10054 The Necklace}

\begin{code}
vector<int> tour(const AList& g_, int start) {
    AList g(g_);
    int n = g.size();
    vector<int> route;
    int odds = 0;
    REP(i, n)
        if (g[i].size()%2 != 0)
            odds++;
    if (odds == 0 || (odds == 2 && g[start].size()%2 == 1))
        dfs(g, start, route);
    reverse(route.begin(), route.end());
    return route;
}
void dfs(AList& g, int a, vector<int>& route) {
    while(!g[a].empty()) {
        int b = g[a].back();
        g[a].pop_back();
        dfs(g, b, route);
    }
    route.push_back(a);
}
\end{code}


\section{�_�C�i�~�b�N�v��@}


\subsection{����Z�[���X�}�����}

\accepted{UVA10496 Collecting Beepers}

$O(n^2 \cdot 2^n)$ �B

\begin{code}
Weight adj[N][N];
Weight tsp[1<<N][N];
REP(i, 1<<n) REP(j, n) tsp[i][j] = WEIGHT_INFTY;
tsp[1<<s][s] = 0; // s����n�܂�path
REP(i, 1<<n) REP(j, n) if (i&(1<<j)) REP(k, n)
    tsp[i|(1<<k)][k] <?= tsp[i][j] + adj[j][k];
Weight* last = tsp[(1<<n)-1];
REP(j, n) last[j] += adj[j][s]; // s�ɖ߂�ꍇ
return *min_element(last, last+n);
\end{code}


\subsection{�i�b�v�T�b�N���}

\accepted{UVA10130 Supersale}

�A�C�e�����������̏ꍇ�̓e�[�u�����������ɑ����B

\begin{code}
Weight v[C+1];
REP(i, c+1) v[i] = 0;
REP(k, n) for(int i = c; i >= costs[k]; i--)
    v[i] >?= v[i - costs[k]] + values[k];
return *max_element(v, v+c+1);
\end{code}


\subsection{�Œ����ʕ����� (LCS)}

\accepted{UVA10192 Vacation}

\begin{code}
int lcs[N+1][M+1];
REP(i, n+1) REP(j, m+1) lcs[i][j] = 0;
REP(i, n) REP(j, m)
    lcs[i+1][j+1] = (str1[i] == str2[j] ? lcs[i][j]+1 : 0)
                    >? lcs[i+1][j] >? lcs[i][j+1];
return lcs[n][m];
\end{code}


\subsection{�Œ����������� (LIS)}

\accepted{UVA231 Testing the CATCHER}

$O(n \log n)$ �Bstrict�ȑ�����̏ꍇ��lower\_bound�ɂ���B

\begin{code}
int lis[N];
REP(i, n) lis[i] = INF;
REP(i, n) *upper_bound(lis, lis+n, v[i]) = v[i];
return find(lis, lis+n, INF) - lis;
\end{code}



\section{��}

\nottested

\subsection{��b}
\begin{code}
typedef complex<double> P;
typedef const P &CP;
namespace std {
    inline bool operator<(const P& a, const P& b) {
        if (NE(a.real(), b.real()))
            return LT(a.real(), b.real());
        return LT(a.imag(), b.imag());
    }
}
struct L { // �����A�������A����
    P pos, dir; // pos��pos+dir�����Ԑ���
};
inline double inp(const P& a, const P& b) { // ���ς����߂�
    return (conj(a)*b).real();
}
inline double outp(const P& a, const P& b) { // �O�ς����߂�
    return (conj(a)*b).imag();
}
inline P proj(const P& p, const P& b) { // p��b�Ɏˉe
    return b*inp(p,b)/norm(b);
}
inline P perf(const L& l, const P& p) { // �����̑�
    L m = {l.pos - p, l.dir};
    return (p + (m.pos - proj(m.pos, m.dir)));
}
inline L proj(const L& s, const L& b) { // �����𒼐��Ɏˉe
    return (L){perf(b, s.pos), proj(s.dir, b.dir)};
}
inline int ccw(const P& p, const P& r, const P& s) {
    P a(r-p), b(s-p);
    int sgn = SGN(outp(a, b));
    if (sgn != 0)
        return sgn;
    if (LT(a.real()*b.real(), 0) || LT(a.imag()*b.imag(), 0))
        return -1;
    if (LT(norm(a), norm(b)))
        return 1;
    return 0;
}
L normalize_line(const L& l) {
    return (L){perf(l, P(0, 0)),
               ( (LT(l.dir.imag(), 0) || (EQ(l.dir.imag(), 0) && LT(l.dir.real(), 0)) )
                    ? -l.dir : l.dir ) / abs(l.dir) };
}
\end{code}


\subsection{�ʐ�}

\begin{code}
double area(vector<P>& g) {
    double s = 0.0;
    REP(i, g.size())
        s += outp(g[i], g[(i+1)%n])/2;
    return abs(s);
}
\end{code}


\subsection{����}

��������Point, Segment, Line��\���B

\subsubsection{����}
\begin{code}
bool ll_intersects(const L& l, const L& m) {
    return NE(abs(normalize_line(l).dir - normalize_line(m).dir), 0);
}
bool ls_intersects(const L& l, const L& s) {
    return (ccw(l.pos, l.pos+l.dir, s.pos)
            *ccw(l.pos, l.pos+l.dir, s.pos+s.dir) <= 0);
}
bool ss_intersects(const L& s, const L& t) {
    return (ls_intersects(s, t) &&
            ls_intersects(t, s));
}
bool line_parallel(const L& l, const L& m) {
    return !ll_intersects(l, m);
}
bool line_equals(const L& l, const L& m) {
    L a = normalize_line(l);
    L b = normalize_line(m);
    return (EQ(abs(a.pos-b.pos), 0) && EQ(abs(a.dir-b.dir), 0));
}
bool sp_intersects(const L& s, const P& p) {
    P r = (p-s.pos)/s.dir;
    return (EQ(r.imag(), 0) && LE(0, r.real()) && LE(r.real(), 1));
}
bool gs_intersects(const G& g, const L& s) {
    int n = g.size();
    for(int i = 0; i < n; i++) {
        int j = (i+1)%n;
        L t(g[i], g[j]-g[i]);
        if (ss_intersects(s, t))
            return true;
    }
    return false;
}
\end{code}

\subsubsection{��_}
\begin{code}
P line_cross(const L& l, const L& m) {
    double num = outp(m.dir, m.pos-l.pos);
    double denom = outp(m.dir, l.dir);
    if (EQ(denom, 0))
        throw 0; // ���L�_����‚ł͂Ȃ�
    return P(l.pos + l.dir*num/denom);
}
\end{code}

\subsubsection{����}
\begin{code}
double pp_distance(const P& a, const P& b) {
    return abs(a - b);
}
double lp_distance(const L& l, const P& p) {
    return abs(perf(l, p) - p);
}
double sp_distance(const L& s, const P& p) {
    const P r = perf(s, p);
    const double pos = ((r-s.pos)/s.dir).real();
    if (LE(0, pos) && LE(pos, 1))
        return abs(r - p);
    return min(abs(s.pos - p),
               abs(s.pos+s.dir - p));
}
double ss_distance(const L& s, const L& t) {
    if (ss_intersects(s, t))
        return 0;
    return min(sp_distance(s, t.pos),
           min(sp_distance(s, t.pos+t.dir),
           min(sp_distance(t, s.pos),
               sp_distance(t, s.pos+s.dir))));
}
double ls_distance(const L& l, const L& s) {
    if (ls_intersects(l, s))
        return 0;
    return min(lp_distance(l, s.pos),
               lp_distance(l, s.pos+s.dir));
}
double ll_distance(const L& l, const L& m) {
    L a = normalize_line(l);
    L b = normalize_line(m);
    if (NE(abs(a.dir-b.dir), 0))
        return 0;
    return abs(a.pos - b.pos);
}
\end{code}

\subsubsection{���}
\begin{code}
bool gp_contains(const G& g, const P& p) {
    double sum = 0.0;
    int n = g.size();
    for(int i = 0; i < n; i++) {
        int j = (i+1)%n;
        if (sp_intersects(L(g[i], g[j]-g[i]), p))
            return true;
        sum += arg((g[j]-p)/(g[i]-p));
    }
    return (abs(sum) > 1);
}
bool gs_contains(const G& g, const L& l) {
    return (gp_contains(g, l.pos) &&
            gp_contains(g, l.pos+l.dir) &&
            !gs_intersects(g, l));
}
bool cgs_contains(const G& g, const P& p) { // For convex polygons
    int n = g.size();
    int sign = 0;
    for(int i = 0; i < n; i++) {
        int j = (i+1)%n;
        int s = SGN(outp(g[j]-g[i], p-g[i]));
        if (sign*s < 0)
            return false;
        if (s == 0 && ccw(g[i], g[j], p) == 0)
            return true;
        sign += s;
    }
    return true;
}
\end{code}


\subsection{�ʕ�}

\derived{IHI Library}

\begin{code}
struct polar_less { // comparator
    Point pivot;
    polar_less(const Point& pivot) : pivot(pivot) {}
    bool operator()(const Point& a, const Point& b) {
        Point z = (a - pivot)/(b - pivot);
        if (NZ(arg(z)))
            return (arg(z) < 0);
        return (abs(z) < 1);
    }
};
void convex_hull(Polygon& polygon, const Polygon& vertices) {
    const int n = vertices.size();
    assert(n >= 3);
    polygon = vertices;
    swap(polygon[0], *min_element(polygon.begin(), polygon.end()));
    Point pivot = polygon[0];
    sort(polygon.begin()+1, polygon.end(), polar_less(pivot));
    {
        int tail = n - 1;
        while(tail > 1 &&
              EQ(arg(polygon[tail-1] - pivot),
                 arg(polygon[n-1] - pivot)))
            tail--;
        reverse(polygon.begin()+tail, polygon.end());
    }
    int m = 3;
    for(int i = 3; i < n; i++) {
        Point pt = polygon[i];
        while(ccw(polygon[m-2], polygon[m-1], pt) <= 0)
            m--;
        polygon[m++] = pt;
    }
    polygon.resize(m);
}
\end{code}

\subsection{�_�C�X�̉�]}

\todo{���������I}




\section{数論}

\subsection{数表}

\paragraph{約数の個数} 

% BEGIN TABULAR MODE
\begin{tabular}{|l|l|}
\hline
N & N以下の数の約数の個数のmax \\ \hline
1e3 & 32 \\ \hline
1e4 & 64 \\ \hline
1e5 & 128 \\ \hline
1e6 & 240 \\ \hline
1e7 & 448 \\ \hline
1e8 & 768 \\ \hline
1e9 & 1344 \\ \hline
INT\_MAX & 1600 \\ \hline
UINT\_MAX & 1920 \\ \hline
\end{tabular}



\paragraph{素数の個数} 

% BEGIN TABULAR MODE
\begin{tabular}{|l|l|}
\hline
N & N以下の素数の個数 \\ \hline
1e1 & 4 \\ \hline
1e2 & 25 \\ \hline
1e3 & 168 \\ \hline
1e4 & 1,229 \\ \hline
1e5 & 9,592 \\ \hline
1e6 & 78,498 \\ \hline
1e7 & 664,579 \\ \hline
1e8 & 5,761,455 \\ \hline
1e9 & 50,847,534 \\ \hline
1e10 & 455,052,511 \\ \hline

\end{tabular}
% END TABULAR MODE


\paragraph{分割数} 

% BEGIN TABULAR MODE
\begin{tabular}{|l|l|}
\hline
N & Nの分割数 \\ \hline
10 & 42 \\ \hline
20 & 627 \\ \hline
30 & 5,604 \\ \hline
40 & 37,338 \\ \hline
50 & 204,226 \\ \hline
60 & 966,467 \\ \hline
70 & 4,087,968 \\ \hline
80 & 15,796,476 \\ \hline
90 & 56,634,173 \\ \hline
100 & 190,569,292 \\ \hline

\end{tabular}


\subsection{定理}

\paragraph{Bertrandの仮説} 

Chebyshevが証明してるので、定理である。任意の自然数nに対してnと2nの間に
必ず素数が存在する。


\paragraph{Wilsonの定理} 

$p$が素数ならば、$(p-1)! \equiv -1 (\mod p)$. 
$p>1$に対しては逆も成り立つ。


\paragraph{Ptolemyの定理} 

円に内接する四角形の対角線の長さの積 = 向かい合う二組の辺の長さの積の和


\paragraph{五心の公式} 

\begin{itemize}
\item{三点が与えられたときの五心の座標}
\item{重心}
\begin{itemize}
\item{(a+b+c)/3}
\end{itemize}
\item{垂心 (TODO: complexで書き直す)}
\end{itemize}
\begin{lstlisting}
// (a,b),(c,d),(e,f)
a-=e,b-=f,c-=e,d-=f;
p=(b*d+a*c)/(a*d-b*c);
p*(d-b)+e,p*(a-c)+f;
\end{lstlisting}
\begin{itemize}
\item{TODO: 残り三つ(傍心はパス?)}
\item{外接円の半径}
\end{itemize}
\begin{lstlisting}
// (a,b),(c,d),(e,f)
hypot(a-=c,b-=d)*hypot(c-=e,d-=f)*hypot(a+c,b+d)*acos(-1)/fabs(a*d-b*c)
\end{lstlisting}
\begin{itemize}
\item{外接円の半径(complex) (未検証)}
\end{itemize}
\begin{lstlisting}
// a,b,c
norm(a-b)*norm(b-c)*norm(c-a)*acos(-1)/fabs(outp(a,b)+outp(b,c)+outp(c,a))
\end{lstlisting}


\paragraph{Pickの定理} 

格子点上に頂点を持つ多角形の、面積S、内部の格子点の数i、辺上の格子点の数bに対して、S=i+b/2-1





\subsection{オイラーの$\phi$関数}

$\phi(n)$は自然数$n$と互いに素な$n$以下の数の個数

\begin{lstlisting}
int phi(int n) {
  int res = n;
  for(int i = 2; i*i <= n; i++) {
    if (n % i == 0) {
      res -= res / i;
      while(n % i == 0)
        n /= i;
    }
  }
  if (n > 1) // n is prime
    res -= res / n;
  return res;
}
 
// phi(1)..phi(N) を求める
void phi_all(int N) {
  int a[N+1], b[N+1];
  REP(i, N+1) {
    a[i] = 1;
    b[i] = i;
  }
 
  REP(k, N+1) {
    if (b[k] < 2)
      continue;
    for(int n = k; n <= N; n += k) {
      for(int m = k-1; b[n]%k == 0; m = k) {
        a[n] *= m;
        b[n] /= k;
      }
    }
  }
  // a[n] == phi(n), b[n] == 1
}
\end{lstlisting}

\subsection{中国剰余定理1}

中国剰余定理とは、$n$個の整数 $m_0, \dots, m_{n-1}$ がどの2つの要素も互
いに素ならば、与えられる $r_0, \dots, r_{n-1}$ に対して
 $x = r_i \mod m_i$ を満たすような$x$が、$\prod_i m_i$を法として唯一つ存在する、というもの。
そのような$x$を $ 0 \leq x \leq \prod_i m_i$ の範囲で求めるアルゴリズム。
但し実装時にはオーバーフローに注意すること。
\w{m[i]はどの2つの要素も互いに素であること。}

\begin{lstlisting}
integer chinese_remainder(const vector<int>& m, const vector<int>& r) {
  int n = m.size();
  integer prod = 1;
  REP(i, n)
  prod *= m[i];

  integer res = 0;
  REP(i, n){
  integer M = prod / m[i];
  integer a = divide(M, 1, m[i]);
  integer R = r[i] - r[i] / prod * prod;
  if(R < 0)
    R += prod;
  res = (res + M * a * r[i] % prod) % prod;
  }
  
  return res;
}
\end{lstlisting}



\subsection{中国剰余定理2}

2つの制約「K=ax+b」「K=cy+d」に対して、それらを同時に満たすKは「K=ez+f」
という一つの式で表せる(か、もしくはそのようなKは存在しない)。
ただし、e=lcm(a,c)。aとcは互いに素でなくてもよい。

以下はそのようなe,fを求めるアルゴリズム。fの値の正規化方法は臨機応変に。
オーバーフローに注意すること。

N個の制約について求めたい場合はN-1回呼んでください。
もしくは、1x+0に対してN回fold。

\w{互いに素である必要はないが、答えがない場合に注意}

\begin{lstlisting}
// ax+bのaとb
struct Constraint {
  int mult;
  int base;
};
 
// BE CAREFUL OF OVERFLOW!!
Constraint chinese_remainder(const Constraint& a, const Constraint& b) {
  Constraint r;
  int g = gcd(a.mult, b.mult);
  int d = b.base - a.base;
 
  // 解がない場合はこのassertで落ちるので、マズい場合は適宜修正すること
  assert(d % g == 0);
  d /= g;
 
  // このへんでオーバーフローに注意
  r.mult = a.mult / g * b.mult;
  r.base = a.mult * d * inv(a.mult/g, b.mult/g) + a.base;
 
  // ここから解の正規化
  // 以下のコードではK=ax+b=cy+d=ez+fに対して、
  // x>=0,y>=0の条件がついていた場合にz>=0を必要十分とするようなfを構成している
  // whileループが長引きそうなら割り算にしたほうがいいかもね
  r.base %= r.mult;
  while(r.base < a.base || r.base < b.base)
    r.base += r.mult;
 
  return r;
}
\end{lstlisting}


\subsection{逆数}

$ a x \equiv 1 (\mod p)$ なる$x$を求める。
$p$が素数でない場合、$a$の逆元が存在すればそれを返す。$a$に0を指定したり、
逆元が存在しなければdivision by zeroで落ちる。

\begin{lstlisting}
int inv(int a, int p) {
  return ( a == 1 ? 1 : (1 - p*inv(p%a, a)) / a + p );
}
\end{lstlisting}


\subsection{拡張ユークリッド互除法}

整数$a, b$に対して、$ax + by = \mathrm{gcd}(a, b)$ となる整数$x, y$を求める。

\begin{lstlisting}
void xgcd(integer a, integer b, integer& x, integer& y) {
  if (b == 0) {
    x = 1;
    y = 0;
  }
  else {
    xgcd(b, a%b, y, x);
    y -= a/b*x;
  }
}
\end{lstlisting}


\subsection{線形ディオファントス方程式}

$an = b \mod m$ となる、非負でかつ最小の$n$を求める。
もしくは、$m$を法とした剰余環上で$b/a$を計算する。
\w{$a, b, m$は非負であること。}

\accepted{UVA700 Date Bugs}

\begin{lstlisting}
struct no_solution {};
integer divide(integer a, integer b, integer m) {
  integer g = gcd(a, m);
  if (b%g != 0)
    throw no_solution();
  integer x, y;
  xgcd(a, m, x, y);
  assert(a*x+m*y == gcd(a,m));
  integer n = x*b/g;
  integer dn = m/g;
  n -= n/dn*dn;
  if (n < 0)
    n += dn;
  return n;
}
\end{lstlisting}



\subsection{行列演算}

\begin{lstlisting}
typedef double* vector_t;
typedef vector_t* matrix_t;
matrix_t new_matrix(int n) {
  matrix_t x = new vector_t[n];
  for(int i = 0; i < n; i++)
    x[i] = new double[n];
  return x;
}
matrix_t dup_matrix(matrix_t x_, int n) {
  matrix_t x = new_matrix(n);
  for(int i = 0; i < n; i++)
    copy(x_[i], x_[i]+n, x[i]);
  return x;
}
void delete_matrix(matrix_t x, int n) {
  for(int i = 0; i < n; i++)
    delete[] x[i];
  delete[] x;
}
matrix_t multiply(matrix_t a, matrix_t b, int n) {
  matrix_t r = new_matrix(n);
  for(int i = 0; i < n; i++) {
    for(int j = 0; j < n; j++) {
      r[i][j] = 0;
      for(int k = 0; k < n; k++)
        r[i][j] += a[i][k] * b[k][j];
    }
  }
  return r;
}
\end{lstlisting}

\subsection{行列の高速べき乗}

\begin{lstlisting}
matrix_t pow(matrix_t& e, int n, int m) {
  matrix_t r = new_matrix(n);
 
  for(int i = 0; i < n; i++)
    for(int j = 0; j < n; j++)
      r[i][j] = ((m&1) == 0 ? (i == j ? 1 : 0) : e[i][j]);
 
  if (m >= 2) {
    matrix_t u = pow(e, n, m/2);
    matrix_t uu = multiply(u, u, n);
    matrix_t z = multiply(r, uu, n);
    delete_matrix(u, n);
    delete_matrix(uu, n);
    delete_matrix(r, n);
    r = z;
  }
 
  return r;
}
\end{lstlisting}


\subsection{ガウスの消去法}

連立方程式 $ Ax=b $ を解く。$A, b$は破壊され、答えが$b$に代入される。
invert, moduloを別途定義すること。
$A, b$はmoduloによる正規化が既に行われているものとする。

\begin{lstlisting}
void gauss(matrix_t& A, vector_t& b, int n, int m) {
  int pi = 0, pj = 0;
  while(pi < n && pj < m) {
    for(int i = pi+1; i < n; i++) {
      if (abs(A[i][pj]) > abs(A[pi][pj])) {
        swap(A[i], A[pi]);
        swap(b[i], b[pi]);
      }
    }
    if (abs(A[pi][pj]) > 0) {
      int d = invert(A[pi][pj]);
      REP(j, m)
        A[pi][j] = modulo(A[pi][j] * d);
      b[pi] = modulo(b[pi] * d);
      for(int i = pi+1; i < n; i++) {
        int k = A[i][pj];
        REP(j, m)
          A[i][j] = modulo(A[i][j] - k * A[pi][j]);
        b[i] = modulo(b[i] - k * b[pi]);
      }
      pi++;
    }
    pj++;
  }
  for(int i = pi; i < n; i++)
    if (abs(b[i]) > 0)
      throw Inconsistent();
  if (pi < m || pj < m)
    throw Ambiguous();
  for(int j = m-1; j >= 0; j--)
    REP(i, j)
      b[i] = modulo(b[i] - b[j] * A[i][j]);
}
\end{lstlisting}


\subsection{LU分解}

aを破壊しLU形式に変換する。pは行交換の情報を保持する。

\begin{lstlisting}
bool lu_decompose(matrix_t& a, int* p, int n) {
  for(int i = 0; i < n; i++)
    p[i] = i;
  for(int k = 0; k < n; k++) {
    int pivot = k;
    for(int i = k+1; i < n; i++)
      if (abs(a[i][k]) > abs(a[pivot][k]))
        pivot = i;
    swap(a[k], a[pivot]);
    swap(p[k], p[pivot]);
    if (abs(a[k][k]) < EP)
      return false;
    for(int i = k+1; i < n; i++) {
      double m = (a[i][k] /= a[k][k]);
      for(int j = k+1; j < n; j++)
        a[i][j] -= a[k][j] * m;
    }
  }
  return true;
}
void lu_solve(matrix_t& a, int* p, vector_t& b, vector_t& x, int n) {
  for(int i = 0; i < n; i++)
    x[i] = b[p[i]];
  for(int k = 0; k < n; k++)
    for(int i = 0; i < k; i++)
      x[k] -= a[k][i] * x[i];
  for(int k = n-1; k >= 0; k--) {
    for(int i = k+1; i < n; i++)
      x[k] -= a[k][i] * x[i];
    x[k] /= a[k][k];
  }
}
\end{lstlisting}


\subsection{単体法}

$\mathrm{minimize} ~ \vec{c} \vec{x} ~~ \mathrm{s.t.} ~~ A \vec{x} = \vec{b} $.
解が存在しない/最適値が発散する場合は \verb|vector_t()| を返す。
単体法は最悪の場合で指数時間がかかるので、あくまで最終手段。これを使う前に、ほかの方法が使えないか十分考えること。
特に、2変数の不等式制約は二次元の凸多角形クリッピングとして捉えることができる。

\begin{lstlisting}
const double INF = numeric_limits<double>::infinity();
typedef vector<double> vector_t;
typedef vector<vector_t> matrix_t;

vector_t simplex(matrix_t A, vector_t b, vector_t c) {
  const int n = c.size(), m = b.size();
 
  REP(i, m) if (b[i] < 0) {
    REP(j, n)
      A[i][j] *= -1;
    b[i] *= -1;
  }
  vector<int> bx(m), nx(n);
  REP(i, m)
    bx[i] = n+i;
  REP(i, n)
    nx[i] = i;
  A.resize(m+2);
  REP(i, m+2)
    A[i].resize(n+m, 0);
  REP(i, m)
    A[i][n+i] = 1;
  REP(i, m) REP(j, n)
    A[m][j] += A[i][j];
  b.push_back(accumulate(ALLOF(b), (double)0.0));
  REP(j, n)
    A[m+1][j] = -c[j];
  REP(i, m)
    A[m+1][n+i] = -INF;
  b.push_back(0);

  REP(phase, 2) {
    for(;;) {
      int ni = -1;
      REP(i, n)
        if (A[m][nx[i]] > EPS && (ni < 0 || nx[i] < nx[ni]))
          ni = i;
      if (ni < 0)
        break;
      int nv = nx[ni];
      vector_t bound(m);
      REP(i, m)
        bound[i] = (A[i][nv] < EPS ? INF : b[i] / A[i][nv]);
      if (!(*min_element(ALLOF(bound)) < INF))
        return vector_t(); // -infinity
      int bi = 0;
      REP(i, m)
        if (bound[i] < bound[bi]-EPS || (bound[i] < bound[bi]+EPS && bx[i] < bx[bi]))
          bi = i;
      double pd = A[bi][nv];
      REP(j, n+m)
        A[bi][j] /= pd;
      b[bi] /= pd;
      REP(i, m+2) if (i != bi) {
        double pn = A[i][nv];
        REP(j, n+m)
          A[i][j] -= A[bi][j] * pn;
        b[i] -= b[bi] * pn;
      }
      swap(nx[ni], bx[bi]);
    }
    if (phase == 0 && abs(b[m]) > EPS)
      return vector_t(); // no solution
    A[m].swap(A[m+1]);
    swap(b[m], b[m+1]);
  }
   vector_t x(n+m, 0);
  REP(i, m)
    x[bx[i]] = b[i];
  x.resize(n);
  return x;
}
\end{lstlisting}

\newpage

\section{その他}

\subsection{ビット演算}

\begin{lstlisting}
unsigned int __builtin_popcount(unsigned int);  // 1であるビットの数を数える。
unsigned long long __builtin_popcountll(unsigned long long);
unsigned int __builtin_ctz(unsigned int);    // 末尾の0であるビットの数を数える。
unsigned long long __builtin_ctzll(unsigned long long);
y = x&-x;   // 最右ビットを抜き出す。
y = x&(x-1); // 最右ビットを落とす。
#define FOR_SUBSET(b, a) for(int b = (a)&-(a); b != 0; b = (((b|~(a))+1)&(a)))
#define FOR_SUBSET(b, a) for(int b = a; b != 0; b = (b-1)&a)
int next_combination(int p) {
  int lsb = p&-p;
  int rem = p+lsb;
  int rit = rem&~p;
  return rem|(((rit/lsb)>>1)-1);
}
\end{lstlisting}


\subsection{平衡木}

Treap. \w{サブツリーに対する操作を行ってはいけない。}

\begin{lstlisting}
template<class Key, class Value, class Cache>
struct Treap {
  Key key;
  Value value;
  int prio;
  Treap* ch[2];
  bool cached;
  Cache cache;
  Treap(const Key& key, const Value& value) : key(key), value(value), prio(rand()), cached(false) {
    ch[0] = ch[1] = 0; // is this necessary?
  }
  Treap* insert(const Key& newkey, const Value& newvalue) {
    if (!this)
      return new Treap(newkey, newvalue);
    if (newkey == key)
      return this;
    int side = (newkey < key ? 0 : 1);
    ch[side] = ch[side]->insert(newkey, newvalue);
    cached = false;
    return rotate(side);
  }
  Treap* rotate(int side) {
    if (ch[side] && ch[side]->prio > prio) {
      Treap* rot = ch[side];
      this->ch[side] = rot->ch[side^1];
      rot->ch[side^1] = this;
      this->cached = rot->cached = false;
      return rot;
    }
    return this;
  }
  void clear() {
    if (this) {
      ch[0]->clear();
      ch[1]->clear();
      delete this;
    }
  }
  Treap* remove(const Key& oldkey) {
    if (!this)
      return this;
    if (key == oldkey) {
      if (!ch[1]) {
        Treap* res = ch[0];
        delete this;
        return res;
      }
      pair<Treap*, Treap*> res = ch[1]->delete_min();
      res.first->ch[0] = this->ch[0];
      res.first->ch[1] = res.second;
      res.first->cached = false;
      delete this;
      return res.first->balance();
    }
    int side = (oldkey < key ? 0 : 1);
    ch[side] = ch[side]->remove(oldkey);
    cached = false;
    return this;
  }
  pair<Treap*, Treap*> delete_min() {
    cached = false;
    if (!ch[0])
      return make_pair(this, ch[1]);
    pair<Treap*, Treap*> res = ch[0]->delete_min();
    ch[0] = res.second;
    return make_pair(res.first, this);
  }
  inline int priority() {
    return (this ? prio : -1);
  }
  Treap* balance() {
    int side = (ch[0]->priority() > ch[1]->priority() ? 0 : 1);
    Treap* rot = rotate(side);
    if (rot == this)
      return this;
    return rot->balance();
  }
  Cache eval() {
    if (!this)
      return Cache();
    if (!cached)
      cache = Cache(key, value, ch[0]->eval(), ch[1]->eval());
    cached = true;
    return cache;
  }
};

// 小さいほうからn番目の要素をクエリできるようにする場合の例
struct SizeCache {
  int size;
  SizeCache() : size(0) {}
  template<class Key, class Value>
  SizeCache(const Key& key, const Value& value, const SizeCache& left, const SizeCache& right)
   : size(left.size+right.size+1) {}
};
template<class Key, class Value>
pair<Key, Value> nth(Treap<Key, Value, SizeCache>* root, int k) {
  int l = root->ch[0]->eval().size;
  if (k < l)
    return nth<Key, Value>(root->ch[0], k);
  if (k == l)
    return make_pair(root->key, root->value);
  return nth<Key, Value>(root->ch[1], k-(l+1));
}
\end{lstlisting}


\subsection{Fenwick Tree}

配列のpartial sumの取得と要素の書き換えをそれぞれ対数時間で行うデータ構造。別名Binary Indexed Tree。
n次元にも容易に拡張できる。
和の取得・要素の更新ともに$\mathrm{O}(\log n)$.

\begin{lstlisting}
T bitquery(const vector<T>& bit, int from, int to) { // [from, to)
  if (from > 0)
    return bitquery(bit, 0, to) - bitquery(bit, 0, from);
  T res = T();
  for(int k = to-1; k >= 0; k = (k & (k+1)) - 1)
    res += bit[k];
  return res;
}

void bitupdate(vector<T>& bit, int pos, const T& delta) {
  for(const int n = bit.size(); pos < n; pos |= pos+1)
    bit[pos] += delta;
}
\end{lstlisting}

\subsection{Range Minimum Query}

列のある区間の最小値を対数時間で求めるデータ構造。
\w{配列サイズは2のべき乗であること。}
初期化に$\mathrm{O}(n)$. 更新・クエリに$\mathrm{O}(\log n)$.

\begin{lstlisting}
// 配列を拡張してRMQに対応させる
void rmq_ext(vector<T>& v) {
  int n = v.size();
  assert(__builtin_popcount(n) == 1); // 長さは2のべき乗でなければならない
  v.resize(n*2);
  for(int i = n; i < 2*n; i++)
    v[i] = min(v[(i-n)*2+0], v[(i-n)*2+1]);
}
// 列の要素を書き換える
void rmq_update(vector<T>& rmq, int pos, const T& value) {
  int n = rmq.size() / 2;
  rmq[pos] = value;
  while(pos < 2*n-1) {
    rmq[pos/2+n] = min(rmq[pos], rmq[pos^1]);
    pos = pos/2+n;
  }
}
// [from, to)の最小値を取り出す
T rmq_query(const vector<T>& rmq, int from, int to) {
  int n = rmq.size() / 2;
  int p = min((from == 0 ? 32 : __builtin_ctz(from)), 31-__builtin_clz(to-from));
  T x = rmq[(from>>p)|((n*2*((1<<p)-1))>>p)];
  from += 1<<p;
  if (from < to)
    x = min(x, rmq_query(rmq, from, to));
  return x;
}
\end{lstlisting}


\subsection{Range Tree}

区間に対する加算、区間中の最小値のクエリ、各要素の値のクエリを対数程度の時間で行う木。
\verb|range_add| に $O((\log n)^2)$.\verb|range_min| に $O(\log n)$.
\verb|query| に $O(\log n)$.

\begin{lstlisting}
struct range_tree {
  int m;
  vector< pair<int, int> > tree;

  range_tree(int n) {
    m = n;
    while(m&(m-1))
      m += m&-m;
    tree.assign(2*m, make_pair(0, 0));
    for(int k = n; k < m; k++)
      tree[k] = make_pair(INF, INF);
    for(int k = m; k < 2*m-1; k++)
      tree[k] = make_pair(0,
                min(tree[((k&~m)<<1)^0].second,
                  tree[((k&~m)<<1)^1].second));
  }
  void range_add(int a, int b, int d) {
    while(a < b) {
      int r = 1, k = a;
      while((a & r) == 0 && a + (r<<1) <= b) {
        r <<= 1;
        k = (k >> 1) | m;
      }
      tree[k].first += d;
      do {
        tree[k].second = tree[k].first +
          (k < m ? zero :
           min(tree[((k&~m)<<1)^0].second,
             tree[((k&~m)<<1)^1].second));
        k = (k >> 1) | m;
      } while(k < 2*m-1);
      a += r;
    }
  }
  int range_min(int a, int b) {
    int res = INF;
    while(a < b) {
      int r = 1, k = a;
      while((a & r) == 0 && a + (r<<1) <= b) {
        r <<= 1;
        k = (k >> 1) | m;
      }
      res = min(res, tree[k].second);
      a += r;
    }
    return res;
  }
  int query(int k) {
    int res = 0;
    while(k < 2*m-1) {
      res += tree[k].first;
      k = (k >> 1) | m;
    }
    return res;
  }
};
\end{lstlisting}



\subsection{α-β枝狩り}

なんだかんだで毎回書くのに苦労するあれ。

\begin{lstlisting}
// [alpha..beta]の値以外を返しても意味を持たない、という意味
int search(int alpha = -INF, int beta = INF) {
  if (finished())
    return 0;
  if (enemy_turn())
    return -search(-beta, -alpha);
  for(move a : possible moves) {
    int p = gain(a);
    alpha >?= p + search(alpha-p, beta-p);
    if (alpha >= beta)
      break;
  }
  return alpha;
}
\end{lstlisting}



\subsection{マトロイド交差}

マトロイド交差の最大独立集合を求める。
\w{M1, M2について独立性判定ができること(M::appendable)。
M2について、独立な集合に要素を加えて非独立になったとき、存在する一意の回路を計算できること(M2::circuit)。}

\begin{lstlisting}
template<class E, class M1, class M2>
set<E> augment(M1 m1, M2 m2, set<E> g, set<E> s) {
  map<E, E> trace;

  vector<E> rights;
  FOR(i, g) {
    if (m1.appendable(s, *i)) {
      rights.push_back(*i);
      trace[*i] = *i;
    }
  }
  while(!rights.empty()) {
    vector<E> lefts;
    REP(i, rights.size()) {
      E e = rights[i];
      if (m2.appendable(s, e)) {
        while(trace[e] != e) {
          s.insert(e); e = trace[e];
          s.erase(e); e = trace[e];
        }
        s.insert(e);
        return s;
      }
      vector<E> c = m2.circuit(s, e);
      REP(j, c.size()) {
        E f = c[j];
        if (trace.count(f) == 0) {
          trace[f] = e;
          lefts.push_back(f);
        }
      }
    }
    rights.clear();
    REP(i, lefts.size()) {
      E f = lefts[i];
      s.erase(f);
      FOR(i, g) {
        E e = *i;
        if (m1.appendable(s, e) && trace.count(e) == 0) {
          trace[e] = f;
          rights.push_back(e);
        }
      }
      s.insert(f);
    }
  }
  return s;
}
\end{lstlisting}



\subsection{Zellerの公式}

\begin{lstlisting}
int zeller(int y, int m, int d) {
  if (m <= 2) {
    y--;
    m += 12;
  }
  return (y + y/4 - y/100 + y/400 + (13 * m + 8) / 5 + d) % 7;
}
\end{lstlisting}


\subsection{Grundy数}

moveできなくなった時に負けるゲームについて、
勝てるかどうかの判定と必勝法の計算を行う道具。
盤面ひとつひとつに数字を割り当てる。後手必勝が0、先手必勝が1以上。
選択はmex。平行していくつかのゲームを進める場合はxor。


\newpage



\section{Tips}

\subsection{STL}

\subsubsection{string}

\begin{code}
// �R���X�g���N�^/���
string();
string(const basic_string& s, size_type pos = 0, size_type n = npos);
string(const char*);
string(const char* s, size_type n);
string(size_type n, char c);
string(InputIterator first, InIter last);

// �}��
void insert(iterator pos, InIter f, InIter l);
string& insert(size_type pos, const string& s);
string& insert(size_type pos, const char* s);
string& append(const string& s);
string& append(size_type n, char c);
string& append(InIter first, InIter last);
void push_back(char c);

// �폜
iterator erase(iterator p);
iterator erase(iterator first, iterator last);
string& erase(size_type pos = 0, size_type n = npos);
void clear();
void resize(size_type n, char c = char());

// �u��
string& replace(size_type pos, size_type n, const string& s);
string& replace(size_type pos, size_type n, size_type n1, char c);
string& replace(iterator first, iterator last, const string& s);
string& replace(iterator first, iterator last, size_type n, char c);
string& replace(iterator first, iterator last, InIter f, InIter l);

// �؂�o��
basic_string substr(size_type pos = 0, size_type n = npos);

// ���� find/rfind
size_type find(const basic_string& s, size_type pos = 0);
size_type find(const char* s, size_type pos, size_type n);
size_type find(const char* s, size_type pos = 0);
size_type find(char c, size_type pos = 0);

// ���� find_{first,last}_{of,not_of}
size_type find_first_of(const basic_string& s, size_type pos = 0);
size_type find_first_of(const char* s, size_type pos, size_type n);
size_type find_first_of(const char* s, size_type pos = 0);
size_type find_first_of(char c, size_type pos = 0);
\end{code}


\newpage

\subsubsection{algorithm}

\paragraph{����}
�@%�S�p���؁[��
\begin{code}
// ���`�T��
InIter find(InIter first, InIter last, const EqualityComparable& value);
ForIter1 find_end(ForIter1 first1, ForIter1 last1, ForIter2 first2, ForIter2 last2, BinPred comp);
InIter find_first_of(InIter first1, InIter last1, ForIter first2, ForIter last2, BinPred comp);
InIter find_if(InIter first, InIter last, Predicate pred);
ForIter max_element(ForIter first, ForIter last, BinPred comp);
ForIter min_element(ForIter first, ForIter last, BinPred comp);
pair<InIter1, InIter2> mismatch(InIter1 first1, InIter1 last1, InIter2 first2, BinPred pred);
ForIter adjacent_find(ForIter first, ForIter last, BinPred pred);

// �񕪒T��
bool binary_search(ForIter first, ForIter last, const T& value, Ordering comp);
pair<ForIter, ForIter> equal_range(ForIter first, ForIter last, const T& value, Ordering comp);
ForIter lower_bound(ForIter first, ForIter last, const T& value, Ordering comp);
ForIter upper_bound(ForIter first, ForIter last, const T& value, Ordering comp);

// �����񌟍�
ForIter1 search(ForIter1 first1, ForIter1 last1, ForIter2 first2, ForIter2 last2, BinPred pred);
ForIter search_n(ForIter first, ForIter last, Integer count, const T& value, BinPred pred);

// �J�E���g
difference_type count(InIter first, InIter last, const EqualityComparable& value);
difference_type count_if(InIter first, InIter last, Predicate pred);
\end{code}


\paragraph{��r}
�@%�S�p���؁[��
\begin{code}
bool equal(InIter1 first1, InIter1 last1, InIter2 first2, BinPred pred);
bool lexicographical_compare(InIter1 first1, InIter1 last1, InIter2 first2, InIter2 last2,
                                 BinPred comp);
\end{code}


\paragraph{�u���E�폜}
�@%�S�p���؁[��
\begin{code}
// �u��
void replace(ForIter first, ForIter last, const T& old_value, const T& new_value);
OutIter replace_copy(InIter first, InIter last, OutIter result,
                         const T& old_value, const T& new_value);
OutIter replace_copy_if(InIter first, InIter last, OutIter result,
                            Predicate pred, const T& new_value) ;
void replace_if(ForIter first, ForIter last, Predicate pred, const T& new_value);

// �폜
ForIter remove(ForIter first, ForIter last, const T& value);
OutIter remove_copy(InIter first, InIter last, OutIter result, const T& value);
OutIter remove_copy_if(InIter first, InIter last, OutIter result, Predicate pred);
ForIter remove_if(ForIter first, ForIter last, Predicate pred);
\end{code}

\paragraph{�\�[�g}
�@%�S�p���؁[��
\begin{code}
void sort(RandIter first, RandIter last, Ordering comp);
void stable_sort(RandIter first, RandIter last, Ordering comp);
bool is_sorted(ForIter first, ForIter last, Ordering comp);
void partial_sort(RandIter first, RandIter middle, RandIter last, Ordering comp);
RandIter partial_sort_copy(InIter first, InIter last, RandIter result_first,
                               RandIter result_last, Compare comp);
void nth_element(RandIter first, RandIter nth, RandIter last, Ordering comp);
\end{code}

\paragraph{����}
�@%�S�p���؁[��
\begin{code}
T accumulate(InIter first, InIter last, T init, BinFunc op);
OutIter adjacent_difference(InIter first, InIter last, OutIter result, BinFunc op);
T inner_product(InIter1 first1, InIter1 last1, InIter2 first2, T init,
                    BinFunc1 op1, BinFunc2 op2);
OutIter partial_sum(InIter first, InIter last, OutIter result, BinOper op);
\end{code}

\paragraph{�C�^���[�V����}
�@%�S�p���؁[��
\begin{code}
UnaFunc for_each(InIter first, InIter last, UnaFunc f);
void generate(ForIter first, ForIter last, Generator gen);
OutIter transform(InIter first, InIter last, OutIter result, UnaFunc op);
OutIter transform(InIter1 first1, InIter1 last1, InIter2 first2, OutIter result, BinFunc op);
\end{code}

\paragraph{�W������}
�@%�S�p���؁[��
\begin{code}
// set operation
bool includes(InIter1 first1, InIter1 last1, InIter2 first2, InIter2 last2, Ordering comp);
OutIter set_difference(InIter1 first1, InIter1 last1, InIter2 first2, InIter2 last2,
                           OutIter result, Ordering comp);
OutIter set_intersection(InIter1 first1, InIter1 last1, InIter2 first2, InIter2 last2,
                             OutIter result, Ordering comp);
OutIter set_symmetric_difference(InIter1 first1, InIter1 last1, InIter2 first2, InIter2 last2,
                                     OutIter result, Ordering comp);
OutIter set_union(InIter1 first1, InIter1 last1, InIter2 first2, InIter2 last2,
                      OutIter result, Ordering comp);

// merge
void inplace_merge(BidirIter first, BidirIter middle, BidirIter last, Ordering comp);
OutIter merge(InIter1 first1, InIter1 last1, InIter2 first2, InIter2 last2,
                  OutIter result, Ordering comp);

// unique
ForIter unique(ForIter first, ForIter last, BinPred pred);
OutIter unique_copy(InIter first, InIter last, OutIter result, BinPred pred);
\end{code}

\paragraph{�q�[�v����}
�@%�S�p���؁[��
\begin{code}
void make_heap(RandIter first, RandIter last, Ordering comp);
void pop_heap(RandIter first, RandIter last, Ordering comp);
void push_heap(RandIter first, RandIter last, Ordering comp);
void sort_heap(RandIter first, RandIter last, Ordering comp);
\end{code}

\paragraph{���̑�}
�@%�S�p���؁[��
\begin{code}
// copy
OutIter copy(InIter first, InIter last, OutIter result);
BidirIter2 copy_backward(BidirIter1 first, BidirIter1 last, BidirIter2
 result);

// fill
void fill(ForIter first, ForIter last, const T& value); 
OutIter fill_n(OutIter first, Size n, const T& value);

// partition
ForIter partition(ForIter first, ForIter last, Predicate pred);
ForIter stable_partition(ForIter first, ForIter last, Predicate pred);

// min/max
const T& max(const T& a, const T& b, BinPred comp);
const T& min(const T& a, const T& b, BinPred comp);

// permutation
bool next_permutation(BidirIter first, BidirIter last, Ordering comp);
bool prev_permutation(BidirIter first, BidirIter last, Ordering comp);

// reverse
void reverse(BidirIter first, BidirIter last);
OutIter reverse_copy(BidirIter first, BidirIter last, OutIter result);

// rotate
ForIter rotate(ForIter first, ForIter middle, ForIter last);
OutIter rotate_copy(ForIter first, ForIter middle, ForIter last, OutIter result);

// swap
void iter_swap(ForIter1 a, ForIter2 b);
void swap(Assignable& a, Assignable& b);
ForIter2 swap_ranges(ForIter1 first1, ForIter1 last1, ForIter2 first2);

// uninitialized
ForIter uninitialized_copy(InIter first, InIter last, ForIter result);
void uninitialized_fill(ForIter first, ForIter last, const T& x);
ForIter uninitialized_fill_n(ForIter first, Size n, const T& x);
\end{code}




\section{���̃��C�u�����ɂ‚���}

���Ƃł����B


\begin{figure}[b]
\begin{center}
\includegraphics[scale=0.4]{fate-aa.eps}

Powered by Fate Testarossa
\end{center}
\end{figure}



\end{document}
