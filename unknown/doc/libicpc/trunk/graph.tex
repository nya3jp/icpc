\section{グラフ}


\subsection{最短路}

\subsubsection{Dijkstra}

一対全最短路問題。非負の重みのみであるときに使用する。

\begin{lstlisting}
Graph g;
vector<int> trace;
 
Weight shortest(int start, int goal) {
  int n = g.size();
  trace.assign(n, -2); // UNREACHABLE
 
  priority_queue<Edge, vector<Edge>, greater<Edge> > q;
  q.push((Edge){-1, start, 0}); // TERMINAL
 
  while(!q.empty()) {
    Edge e = q.top();
    q.pop();
    if (trace[e.dest] >= 0)
      continue;
    trace[e.dest] = e.src;
    if (e.dest == goal)
      return e.weight;
    FOR(it, g[e.dest])
      if (trace[it->dest] == -2)
        q.push((Edge){it->src, it->dest, e.weight + it->weight});
  }
  return -1;
}
 
vector<int> buildRoute(int goal) {
  if (trace[goal] == -2)
    return vector<int>(); // UNREACHABLE
  vector<int> route;
  for(int i = goal; i >= 0; i = trace[i])
    route.push_back(i);
  reverse(route.begin(), route.end());
  return route;
}
\end{lstlisting}

\subsubsection{Bellman-Ford}

一対全最短路問題。負の重みがあるときに使用する。$n$回やっても緩和が
起こるようならnegative cycleが存在する。

枝の重みが非負であるグラフにおいて、一対全最短経路を求めるアルゴリズム。
負の重みを持つ枝が存在するならBellman-Ford法などのアルゴリズムを用いる。
\w{パスは非負の重みであること。}

\begin{lstlisting}
Graph g;
 
bool shortest(int start) {
  int n = g.size();
  vector<Weight> costs(n, WEIGHT_INFTY);
  vector<int> trace(n, -2);
  costs[start] = 0;
  trace[start] = -1;    // TERMINAL
 
  REP(k, n) REP(a, n) if (costs[a] != WEIGHT_INFTY) FOR(it, g[a]) {
    int b = it->dest;
    Weight w = costs[a] + it->weight;
    if (w < costs[b]) {
      costs[b] = w;
      trace[b] = a;
    }
  }
  REP(a, n) FOR(it, g[a]) // negative cycleのチェック
    if (costs[a] + it->weight < costs[it->dest])
      return false;
  return true;
}
\end{lstlisting}

\subsubsection{Warshall-Floyd}

全対全最短路問題。
\w{自身への距離を0とするかどうかに応じて隣接行列の対角成分を
初期化すること。}
道順がいるときは再帰的に。

\begin{lstlisting}
Weight g[N][N];
int trace[N][N];
 
void shortest() {
  REP(i, n) REP(j, n)
    trace[i][j] = -1;
  REP(j, n) REP(i, n) REP(k, n) {
    if (g[i][k] > g[i][j] + g[j][k]) {
      g[i][k] = g[i][j] + g[j][k];
      trace[i][k] = j;
    }
  }
}
void buildRoute(vector<int>& route, int src, int dest, bool rec = false) {
  if (!rec)
    route.clear();
  int inter = trace[src][dest];
  if (inter < 0) {
    route.push_back(src);
  }
  else {
    buildRoute(route, src, inter, true);
    buildRoute(route, inter, dest, true);
  }
  if (!rec)
    route.push_back(dest);
}
\end{lstlisting}


\subsection{最小広域木}

任意の頂点集合分割について、それらの間にある枝のうち最小の重みを
持つ枝を含む最小広域木が存在する。

枝を重みの順でソートしてUnion-FindするのがKruskal。
任意の点から始めて、最小コストで取り込める点を取っていくのがPrim。
Primは隣接行列表現を用いれば$\mathrm{O}(V^3)$で走る。


\begin{lstlisting}
pair<Weight, Edges> prim(Graph& g) {
  int n = g.size();
  priority_queue<Edge, vector<Edge>, greater<Edge> > q;
  vector<bool> visited(n, false);
  Edges tree;
  Weight weight = 0;
 
  q.push((Edge){-1, 0, 0});
  while(!q.empty()) {
    Edge e = q.top();
    q.pop();
    if (visited[e.dest])
      continue;
    visited[e.dest] = true;
    if (e.src >= 0) {
      tree.push_back(e);
      weight += e.weight;
    }
    FOR(it, g[e.dest])
      if (!visited[it->dest])
        q.push(*it);
  }
  return make_pair(weight, tree);
}
\end{lstlisting}

\begin{lstlisting}
pair<Weight, Edges> kruskal(Graph& g) {
  int n = g.size();
  vector<Edge> edges;
  REP(a, n) FOR(it, g[a])
    if (a < it->dest)
      edges.push_back(*it);
  sort(ALLOF(edges));
  UnionFind uf(n);
  Edges tree;
  Weight weight = 0;
  REP(i, edges.size()) {
    if ((int)tree.size() >= n-1)
      break;
    Edge& e = edges[i];
    if (uf.link(e.src, e.dest)) {
      tree.push_back(e);
      weight += e.weight;
    }
  }
  return make_pair(weight, tree);
}
\end{lstlisting}



\subsection{無向オイラー路}

グラフの一筆書き(オイラー路)を求めるアルゴリズム。
無向グラフが辺連結なとき、奇点が存在しない場合無向オイラー閉路が
存在し、奇点が2個だけ存在する場合無向オイラー路が存在する。
\w{グラフgは隣接行列で与えられること。多重辺・セルフループを許す。}


\begin{lstlisting}
int g[N][N];
int adj[N][N];
 
void dfs(int a, vector<int>& route) {
  REP(b, N){
    if(adj[a][b]){
      adj[a][b]--;
      adj[b][a]--;
      dfs(b, route);
    }
  }
  route.push_back(a);
}
 
vector<int> euler(int start) {
  int odd = 0;
  int nEdges = 0;
  REP(i, N){
    int deg = accumulate(g[i], g[i] + N, 0);
    if(deg % 2 == 1)
      odd++;
    nEdges += deg;
  }
  nEdges /= 2;
 
  vector<int> route;
  int startdeg = accumulate(g[start], g[start] + N, 0);
  if(odd == 0 || (odd == 2 && startdeg % 2 == 1)){
    memcpy(adj, g, sizeof(g));
    dfs(start, route);
  }
  
  if((int)route.size() != nEdges + 1) // 非連結だった場合
    route.clear();
  reverse(route.begin(), route.end());
  return route;
}
\end{lstlisting}



\subsection{有向オイラー路}

グラフの一筆書き(オイラー路)を求めるアルゴリズム。
有向グラフが辺連結なとき、全ての点において出次数と入次数が等しい場合有向
オイラー閉路が存在し、出次数1の点と入次数1の点が各1個でそれ以外の全ての
点で出次数と入次数が等しい場合オイラー路が存在する。
\w{グラフは隣接リストで与えられること。多重辺・セルフループを許す。}

\begin{lstlisting}
void dfs(Graph& g, int a, vector<int>& route) {
  while(!g[a].empty()){
    int b = g[a].back();
    g[a].pop_back();
    dfs(g, b, route);
  }
  route.push_back(a);
}
 
vector<int> euler(Graph& g, int start) {
  int nNodes = g.size();
  int nEdges = 0;
  vector<int> deg(nNodes, 0);
  REP(i, nNodes){
    nEdges += g[i].size();
    REP(j, g[i].size())
      deg[g[i][j]]--; // 入次数
    deg[i] += g[i].size(); // 出次数
  }
  
  vector<int> route;
  int nonzero = nNodes - count(deg.begin(), deg.end(), 0);
  if(nonzero == 0 || (nonzero == 2 && deg[start] == 1)) {
    Graph g_(g);
    dfs(g_, start, route);
  }
    
  if((int)route.size() != nEdges + 1) // 非連結だった場合
    route.clear();
  reverse(route.begin(), route.end());
  return route;
}
\end{lstlisting}

\subsection{強連結成分分解}

出力される強連結成分はDAGに縮約したときのトポロジカル順序に従っている。

\begin{lstlisting}
Graph g;
vector<bool> visited;
 
vector< vector<int> > scc() {
  int n = g.size();
  Graph r(n);    // make reversed graph
  REP(a, n) FOR(it, g[a])
    r[it->dest].push_back((Edge){it->dest, it->src});
  vector<int> order;
  visited.assign(n, false);
  REP(i, n) dfs(i, order);
  reverse(order.begin(), order.end());
  g.swap(r);
  vector< vector<int> > components;
  visited.assign(n, false);
  REP(i, n)
    if (!visited[order[i]])
      components.push_back(vector<int>()) ,
        dfs(order[i], components.back());
  g.swap(r);
  return components;
}
void dfs(int a, vector<int>& order) {
  if (visited[a])
    return;
  visited[a] = true;
  FOR(it, g[a]) dfs(it->dest, order);
  order.push_back(a);
}
\end{lstlisting}



\subsection{二重頂点連結成分}

点$a$を除去したときに$m$個の連結成分に分割される時、artsに$m-1$個の数$a$が代入される。
bccsには、二重頂点連結成分を誘導する頂点集合の集合が代入される。
\w{arts, bccsは空にして渡すこと。}

\begin{lstlisting}
int n;
bool g[N][N];

int bcc_decompose(int a, int depth, vector<int>& labels, vector<int>& comp,
                  vector<int>& arts, vector< vector<int> >& bccs) {
    int children = 0, upward = labels[a] = depth;
    REP(b, n) if (g[a][b]) {
        int u = labels[b];
        if (u < 0) {
            int k = comp.size();
            u = bcc_decompose(b, depth+1, labels, comp, arts, bccs);
            if (u >= depth) {
                comp.push_back(a);
                bccs.push_back(vector<int>(comp.begin()+k, comp.end()));
                comp.erase(comp.begin()+k, comp.end());
                if (depth > 0 || children > 0)
                    arts.push_back(a);
            }
        }
        upward <?= u;
        children++;
    }
    comp.push_back(a);
    if (depth == 0 && children == 0)
        bccs.push_back(comp);
    return upward;
}

void bcc_decompose(vector<int>& arts, vector< vector<int> >& bccs) {
    vector<int> comp, labels(n, -1);
    REP(r, n) if (labels[r] < 0) {
        comp.clear();
        bcc_decompose(r, 0, labels, comp, arts, bccs);
    }
}
\end{lstlisting}


\subsection{橋}

\begin{lstlisting}
int bridge(Graph& g, int here, int parent, vector<int>& labels,
       vector<int>& current, vector< vector<int> >& comps, int& id) {
  int& myid = labels[here];
  assert(myid < 0);
  myid = id++;

  int res = myid;
  Edges& v = g[here];
  REP(i, v.size()) {
    int there = v[i];
    if (there == parent)
      continue;
    int child;
    if (labels[there] < 0) {
      child = bridge(g, there, here, labels, critical, id);
      if (child > myid)
        critical[here][there] = critical[there][here] = true;
    }
    else {
      child = labels[there];
    }
    res <?= child;
  }

  return res;
}
\end{lstlisting}



\subsection{Union-Find}

\begin{lstlisting}
int n;
vector<int> data(n, -1);
bool link(int a, int b) { // 新たな併合を行うとtrue
  int ra = find_root(a);
  int rb = find_root(b);
  if (ra != rb) {
    if (data[rb] < data[ra])
      swap(ra, rb);
    data[ra] += data[rb];
    data[rb] = ra;
  }
  return (ra != rb);
}
bool check(int a, int b) { // 同じ集合ならtrue
  return (find_root(a) == find_root(b));
}
int find_root(int a) { // 代表元を返す
  return ((data[a] < 0) ? a : (data[a] = find_root(data[a])));
}
\end{lstlisting}



\subsection{二部グラフのマッチング}

DFSで増大路を求めていく。片側の辺でメモすれば$\mathrm{O}(V(V+E))$.
二部グラフにおいて、最大マッチングの大きさは最小点被覆の大きさに一致する。
以下はもっと速いHopcroft-Karp. 計算量は$\mathrm{O}(\sqrt{V}(V+E))$.

\begin{lstlisting}
vector< vector<int> > g;
int n, m;

vector<bool> visited, matched;
vector<int> levels, matching;

bool augment(int left) {
  if (left == n)
    return true;
  if (visited[left])
    return false;
  visited[left] = true;
  REP(i, g[left].size()) {
    int right = g[left][i];
    int next = matching[right];
    if (levels[next] > levels[left] && augment(next)) {
      matching[right] = left;
      return true;
    }
  }
  return false;
}
int match() {
  matching.assign(m, n);
  matched.assign(n, false);
  bool cont;
  do {
    levels.assign(n+1, -1);
    levels[n] = n;
    queue<int> q;
    REP(left, n) {
      if (!matched[left]) {
        q.push(left);
        levels[left] = 0;
      }
    }
    while(!q.empty()) {
      int left = q.front();
      q.pop();
      REP(i, g[left].size()) {
        int right = g[left][i];
        int next = matching[right];
        if (levels[next] < 0) {
          levels[next] = levels[left] + 1;
          q.push(next);
        }
      }
    }
    visited.assign(n, false);
    cont = false;
    REP(left, n)
      if (!matched[left] && augment(left))
        matched[left] = cont = true;
  } while(cont);
  return count(ALLOF(matched), true);
}
\end{lstlisting}


\subsection{ハンガリアン法}

$\mathrm{O}(n^2 m) ~ (n < m).$

\begin{lstlisting}
#define residue(i,j) (adj[i][j] + ofsleft[i] + ofsright[j])

vector<int> min_cost_match(vector< vector<int> > adj) {
  int n = adj.size();
  int m = adj[0].size();
  vector<int> toright(n, -1), toleft(m, -1);
  vector<int> ofsleft(n, 0), ofsright(m, 0);

  REP(r, n) {
    vector<bool> left(n, false), right(m, false);
    vector<int> trace(m, -1), ptr(m, r);
    left[r] = true;

    for(;;) {
      int d = numeric_limits<int>::max();
      REP(j, m) if (!right[j])
        d <?= residue(ptr[j], j);
      REP(i, n) if (left[i])
        ofsleft[i] -= d;
      REP(j, m) if (right[j])
        ofsright[j] += d;
      int b = -1;
      REP(j, m) if (!right[j] && residue(ptr[j], j) == 0)
        b = j;
      trace[b] = ptr[b];
      int c = toleft[b];
      if (c < 0) {
        while(b >= 0) {
          int a = trace[b];
          int z = toright[a];
          toleft[b] = a;
          toright[a] = b;
          b = z;
        }
        break;
      }
      right[b] = left[c] = true;
      REP(j, m) if (residue(c, j) < residue(ptr[j], j))
        ptr[j] = c;
    }
  }
  return toright;
}
\end{lstlisting}



\subsection{安定結婚}

任意の希望リストについて、安定なマッチングが少なくとも1つ存在することが知られている。
以下は男性優先の解を求める。
返り値は \verb|res[womanID] = manID| なる割り当て。
各人について、優先度の高い相手から順に並べる。
\verb|_man[manID][rank] = womanID|, \verb|_woman[womanID][rank] = manID|.
$\mathrm{O}(n^2).$

\begin{lstlisting}
inline vector<int> stable_marriage(const vector<vector<int> >& _man,
                   const vector<vector<int> >& _woman) {
  int n = _man.size();
  vector<vector<int> > man(n, n), woman(n, n);
  vector<int> res(n, -1);
  
  REP(i, n) REP(j, n){
    man[i][j] = _man[i][n-j-1];
    woman[i][_woman[i][j]] = n - j - 1;
  }
  REP(i, n){
    for(int manID = i; manID >= 0; ){
      int womanID = man[manID].back();
      man[manID].pop_back();
      int prevman = res[womanID];
      if(prevman < 0 || woman[womanID][prevman] < woman[womanID][manID]){
        res[womanID] = manID;
        manID = prevman;
      }
    }
  }
  return res;
}
\end{lstlisting}


\subsection{最大流}

重み無しグラフの意味での最短路に沿ってフローを流し続けるのがEdmonds-Karp。
それにレベルグラフを導入し改善を図ったのが以下のDinic.
計算時間は$\mathrm{O}(V^4)$.
隣接リスト表現を使えば$\mathrm{O}(V^2 E)$.

\begin{lstlisting}
int n, s, t;
int cap[N][N], flow[N][N];
int levels[N];
bool finished[N];
#define residue(i,j) (cap[i][j] - flow[i][j])

void levelize() {
  memset(levels, -1, sizeof(levels));
  queue<int> q;
  levels[s] = 0; q.push(s);
  while(!q.empty()) {
    int here = q.front(); q.pop();
    REP(there, n) if (levels[there] < 0 && residue(here, there) > 0) {
      levels[there] = levels[here] + 1;
      q.push(there);
    }
  }
}
int augment(int here, int cur) {
  if (here == t || cur == 0)
    return cur;
  if (finished[here])
    return 0;
  finished[here] = true;
  REP(there, n) if (levels[there] > levels[here]) {
    int f = augment(there, min(cur, residue(here, there)));
    if (f > 0) {
      flow[here][there] += f;
      flow[there][here] -= f;
      finished[here] = false;
      return f;
    }
  }
  return 0;
}
int maxflow() {
  memset(flow, 0, sizeof(flow));
  int total = 0;
  for(bool cont = true; cont; ) {
    cont = false;
    levelize();
    memset(finished, 0, sizeof(finished));
    for(int f; (f = augment(s, INF)) > 0; cont = true)
      total += f;
  }
  return total;
}
\end{lstlisting}



\subsection{最小カット}

Stoer-Wagner. $\mathrm{O}(V^3)$.
グラフの隣接リスト表現とpriority queueを使えば$\mathrm{O}(VE \log E)$.
\w{非負の重みであること。隣接行列}\verb|adj|\w{は破壊される。}

\begin{lstlisting}
Weight adj[N][N];
int n;

int min_cut() {
  Weight res = WEIGHT_INFTY;

  vector<int> v;
  REP(i, n)
    v.push_back(i);
  for(int m = n; m > 1; m--) {
    vector<Weight> ws(m, 0);
    int s, t;
    Weight w;
    REP(k, m) {
      s = t;
      t = max_element(ws.begin(), ws.end()) - ws.begin();
      w = ws[t];
      ws[t] = -1;
      REP(i, m)
        if (ws[i] >= 0)
          ws[i] += adj[v[t]][v[i]];
    }
    REP(i, m) {
      adj[v[i]][v[s]] += adj[v[i]][v[t]];
      adj[v[s]][v[i]] += adj[v[t]][v[i]];
    }
    v.erase(v.begin()+t);
    res <?= w;
  }
  return res;
}
\end{lstlisting}



\subsection{最小費用流}

Primal-Dual. \w{コスト行列は反対称であること。キャパシティ行列は非負であること。
キャパシティが正の枝はコストが非負であること。}

\begin{lstlisting}
#define residue(i,j) (cap[i][j] - flow[i][j])
#define rcost(i,j) (cost[i][j] + pot[i] - pot[j])
Weight min_cost_flow(vector< vector<Flow> > cap, vector< vector<Weight> > cost, int s, int t) {
  int n = cap.size();

  vector< vector<Flow> > flow(n, n);
  vector<Weight> pot(n);

  Weight res = 0;
  for(;;) {

    vector<Weight> dists(n+1, WEIGHT_INFTY);
    vector<bool> visited(n, false);
    vector<int> trace(n, -1);
    dists[s] = 0;

    for(;;) {
      int cur = n;
      REP(i, n) if (!visited[i] && dists[i] < dists[cur])
        cur = i;
      if (cur == n)
        break;
      visited[cur] = true;
      REP(next, n) if (residue(cur, next) > EPS) {
        if (dists[next] > dists[cur] + rcost(cur, next) && !visited[next]) {
          dists[next] = dists[cur] + rcost(cur, next);
          trace[next] = cur;
        }
      }
    }
    if (!visited[t])
      break;

    Flow f = FLOW_INFTY;
    for(int c = t; c != s; c = trace[c])
      f <?= residue(trace[c], c);
    for(int c = t; c != s; c = trace[c]) {
      flow[trace[c]][c] += f;
      flow[c][trace[c]] -= f;
      res += cost[trace[c]][c] * f;
    }

    REP(i, n)
      pot[i] += dists[i];
  }

  return res;
}
\end{lstlisting}




\subsection{二部グラフの辺彩色}

二部グラフの辺に彩色をし、同一ノードに繋がっている辺が同じ色を持たないようにする。
最小彩色数はグラフの最大次数に等しい。

\begin{lstlisting}
int n, m;          // 左右のノード数
vector< vector<int> > g;  // グラフ (index -> [index])
vector< vector<int> > cl; // 色づけ (index -> color -> index)

BGEC(int n, int m) : n(n), m(m), g(n+m) {}
void add_edge(int a, int b) { // 左a、右bの間に辺を張る
  g[a].push_back(n+b);
  g[n+b].push_back(a);
}
int color() {
  int d = 0; // 彩色数
  REP(i, n+m)
    d = max<int>(d, g[i].size());
  cl.assign(n+m, vector<int>(d, -1));
  REP(a, n) REP(i, g[a].size()) {
    int b = g[a][i];
    int ca = min_element(ALLOF(cl[a])) - cl[a].begin();
    int cb = min_element(ALLOF(cl[b])) - cl[b].begin();
    if (ca != cb) {
      augment(b, ca, cb);
      cb = ca;
    }
    cl[a][ca] = b;
    cl[b][cb] = a;
  }
  return d;
}
void augment(int a, int c1, int c2) {
  int b = cl[a][c1];
  if (b >= 0) {
    augment(b, c2, c1);
    cl[b][c2] = a;
    cl[a][c1] = -1;
  }
  cl[a][c2] = b;
}
\end{lstlisting}



\subsubsection{最小平均長閉路}

% DESCRIPTION:
平均長が最小である単純閉路を探すアルゴリズム。KarpによるBellman-Fordアルゴリズムの変形で、次の事実を用いる:
\begin{equation*}
\mathrm{minavg} = \min_{v \in V} \max_{0 \leq i < n}
 \frac{d_n(v) - d_i(v)}{n - i}.
\end{equation*} %
このコードで求まるのは閉路の平均長。閉路自体を求めるには、Bellman-Fordの過程で最小値に至るルートを保存して後で辿る。
最小費用循環を効率よく求める際に用いる。
$ \mathrm{O}(nm) $.

\begin{lstlisting}
Edges edges;

Weight dp[N+1][N] = { { 0 } };
REP(k, n) {
    REP(i, n)
        dp[k+1][i] = INF;
    REP(i, m) {
        Edge& e = edges[i];
        if (dp[k][e.src] < INF)
            dp[k+1][e.dest] <?= dp[k][e.src] + e.weight;
    }
}
Weight res = INF;
REP(i, n) {
    Weight lo = -INF;
    REP(k, n) if (dp[n][i] < INF && dp[k][i] < INF)
        lo >?= (dp[n][i] - dp[k][i]) / (n - k);
    if (lo > -INF)
        res <?= lo;
}
\end{lstlisting}



\subsection{範囲の更新}

\w{Intervalsには初期値として「負無限大=無色」を代入しておくこと。}

\begin{lstlisting}
typedef map<int, int> Intervals;

// paint [left, right) with color c
void paint(Intervals& v, int left, int right, int c) {
  assert(left < right);
  Intervals::iterator first = v.lower_bound(left), last = v.upper_bound(right);
  int p = (--last)->second;
  last++;
  v.erase(first, last);
  v[left] = c;
  v[right] = p;
}
\end{lstlisting}

\newpage
